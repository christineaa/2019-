% !Mode:: "TeX:UTF-8"
%!TEX program  = xelatex

%\documentclass{cumcmthesis}
\documentclass[withoutpreface,bwprint]{cumcmthesis} %去掉封面与编号页

\usepackage{url}
\title{ }
\tihao{B}
\baominghao{4321}
\schoolname{**大学}
\membera{A}
\memberb{B}
\memberc{C}
\supervisor{老师}
\yearinput{2018}
\monthinput{07}
\dayinput{16}

\begin{document}
	\title{高压油管的压力控制}

	\maketitle

 \begin{abstract}




\keywords{\quad  \quad   \quad  }
\end{abstract}




%目录
%\tableofcontents

%\newpage
\section{问题重述}
%为何研究高压油管的压力控制
燃油喷射系统由喷油泵、高压油管和喷油嘴等组成,是柴油机燃料的供给来源,被认为是柴油机的“心脏”。高压油管是燃油从喷油泵输送到喷油器的重要通道,其正常工作是保证发动机高效率工作的因素之一。高压油管一般采取多次喷射,能有效改善柴油机的排放性和燃油经济性。而燃油多次、间歇性进入和喷出会导致高压油管内产生压力波动,使得喷出的燃油量出现偏差,燃油量供应不足将显著影响柴油机的工作效率。因此,应设计合理的喷油泵与喷油器的工作机制以确保管内压力尽可能稳定。



\subsection{问题的提出}
问题一:对于某一特定的高压油管,通过单向阀开关控制向高压油管供油时间的长短,通过喷油器的开关控制向外的喷油量。问题一从相对简单的燃油进出特性开始考虑,设置单向阀每打开一次后就要关闭10$ms$,喷油嘴每秒工作10次,每次工作时喷油时间为2.4$ms$。在喷油孔周期性工作的情况下,要求设置合理单向阀开启时间以确保高压油管内压力尽可能保持在100MPa;调整开启时长使高压油管内的压力从100MPa通过2$s$、5$s$或10$s$增加到150MPa。


问题二:调整燃油进出特性,考虑进油由高压油泵的柱塞控制,喷油由喷油嘴的针阀控制,喷油嘴与问题一中有相同的工作特性。高压油泵的压油由凸轮驱动,设置合理的凸轮角速度,使高压油管内的压力在整个工作进程中尽量稳定在100MPa。


问题三:考虑较为复杂的情况,增加一个与问题一规律相同的喷油嘴和一个单向减压阀。减压阀可以使高压油管内的燃油在压力下回流到外部低压油路中。要求设计高压油泵与减压阀的控制方案。

\section{模型的假设}
在喷射过程计算中,考虑喷射过程的全部因素是不切实际的,为了研究该具体情况高压油管的喷射过程,做出适当假定,以建立简明的计算模型。
\begin{itemize}
\item 假设单向阀和喷油嘴周期性工作;在每一周期内,工作时间相同。
\item 忽略高压油管接口处的集中容积、油管变截面等对喷射过程的影响。
\item 喷油压力变化所引起的温度变化是微小的\cite{bib:one},因此不考虑温度随压力和时间的变化。
\item 假定燃油在各腔中静止压缩、膨胀,状态变化瞬间到达平衡,在同一时刻,各腔内压强处处相等。
\item 不考虑油管在泵、嘴两端接口处的流动损失

\end{itemize}

\section{符号说明}

\begin{center}
\begin{table}[!ht]
\caption{符号说明}
\centering
\begin{tabular}{cc}
 \toprule[1.5pt]
 \makebox[0.3\textwidth][c]{符号}	&  \makebox[0.4\textwidth][c]{意义} \\ \midrule
 $P_c$ & 高压油管内平均压强 \\ 
 $P_p$ & 高压油泵内的平均压强 \\ 
 $Q_{in}$ & 进入高压油管的流量 \\
 $Q_{out}$ & 喷出高压油管的流量 \\ 
 $Q_{VP}$ &  柱塞腔压力变化所引起的压缩油量变化率\\
 $\rho$ & 燃油的密度 \\
 $E$ & 燃油的弹性模量 \\ 
 $V_c$ & 高压油管内腔容积 \\  \bottomrule[1.5pt]
\end{tabular}
\end{table}
\end{center}

\section{问题分析}
由于燃油是可压缩流体,燃油进入和喷出将会高压油管内的燃油密度,而压强的变化量与密度变化量成正比。为了使压力尽可能稳定在100MPa,根据假设4,即尽量使平均压强变化量为0。因此要通过进出流量、密度和弹性模量建立压强与时间之间的关系。
在问题一中,进入高压油管的流量与A小孔两边的压力有关;喷出的流量由喷油速率示意图可以确定。


在问题二中,由于单向阀的开关不再随时间固定的开闭,而由单向阀两边的压力决定,因此进入高压油管的燃油量由凸轮的轮动有关;喷出的流量由喷油嘴的针阀上下运动决定。凸轮轮动使高压油泵的体积变化,进而影响燃油的密度和压力,当燃油压力大于高压油管内的燃油压力时,单向阀开启,燃油从高压油泵流入高压油管,从而影响高压油管的压力。凸轮转动只由凸轮的角速度决定,因此,应建立模型体现压力与凸轮角速度的关系和时间的关系。


在问题二基础上,增加一个减压单向阀和喷油嘴,只需调整问题二的模型。

 
\section{模型的建立与解决}
\subsection{问题一:阀-管-嘴模型}
在问题一中,根据题目所给条件,只需考虑单向阀流入高压油管的燃油量与喷出的燃油量,而流入与喷出的燃油将影响高压油管内的燃油密度,进而影响管内压力,因此我们将把高压油管、单向阀与喷油嘴看作一个系统,研究该系统内压力的变化,期望建立压力与时间的变化方程,
\subsubsection{公式推导}
根据燃油的可压缩性,在高压油管与外界没有质量交换时,其内的压强变化为
\begin{equation}
	E = -\frac{\mathrm{d}P_c}{\mathrm{d}V}V
\end{equation}
其中:$V$代表高压油管内的体积,符号表示压强随着体积的增大而减小。
当单向阀和喷油嘴开启时,进出高压油管的流量为
\begin{equation}
	Q = CA\sqrt{\frac{2\Delta P}{\rho}}
\end{equation}
其中:$C = 0.85$为流量系数,$A$为小孔的面积,$\Delta P$为小孔两边的压力差,$\rho$为高压侧燃油的密度。

进入高压油管的燃油量与A处的单向闸开启时间有关,假设单向闸在工作周期开始时便开启,设开启时间为$\widetilde{t}$,记工作周期为$T_{in} = 10ms + \widetilde{t}$那么进入高压油管的流量为:
\begin{equation}
	Q_in = \left\{ 
	\begin{array}{ll}
	CA\sqrt{\frac{2(P_c-P_p)}{\rho_c}}&,nT_{in}\leq t <nT_{in} + \widetilde{t} \\
	0&,nT
	-{in}+\widetilde{t}\leq t <(n+1)T_{in}
	\end{array}
	\right.
\end{equation}
$n = 1,2,3,...$

对喷油嘴,假设喷油嘴在一个工作周期开始时便开启,记工作周期为$T_{out} = 10 ms$。由题目所给喷油速率$(mm^3)$与时间$(ms)$的关系可知喷油流量为
\begin{equation}
Q_{in} = \left\{ 
\begin{array}{ll}
100t&,nT_{out}\leq t<nT_{out}+0.2 \\
20&,nT_{out}+0.2\leq t< nT_{out}+2.2\\
-100t+240&,nT_{out}+0.2\leq t< nT_{out}+2.4\\
0&,nT_{out}+2.4\leq t <(n+1)T_{out}
\end{array}
\right.
\end{equation}
$n = 1,2,3,...$

因此,当高压油管与外界有质量交换时压力控制方程(1)结合(3)-(4),得到了高压油管内压力变化量与时间的关系:
\begin{equation}
	\mathrm{d}P_c = \frac{E}{V_c}(Q_{in} - Q_{out})
\end{equation}

在公式(5)中,弹性模量与密度都是与压强有关的,因此要先得到两者与压强的表达式。
\subsubsection{弹性模量与压强的关系}
根据附件三所给的数据,我们可以通过拟合得到弹性模量与压强之间的关系为:
\begin{equation}
E = 1489 e^{0.00284P} + 48.79 e^{0.01376 P}
\end{equation}
拟合误差为


\subsubsection{燃油压强与燃油密度的关系}
燃油的压力变化量与密度变化量成正比,比例系数为$\frac{E}{\rho}$,其中$E$为弹性模量,因此可列出常微分方程:
\begin{equation}
\frac{\mathrm{d}P}{\mathrm{d}\rho} = \frac{E}{\rho} = \frac{1489 e^{0.00284P} + 48.79 e^{0.01376 P}}{\rho}
\end{equation}
已知方程的边值条件为$P = 100MPa, \rho = 0.850\ mh/mm^3$,同时带入公式(6),可解得燃油压强与密度的表达式:
\subsection{问题一模型的解决}
我们的目标是找到适合的$\widetilde{t}$,使微分方程公式(5)的解稳定后的平均值为100$MPa$且方差较小。
\subsubsection{基于四阶龙格库塔的数值仿真}
对于复杂的常微分方程常常难以得到解析解,因此我们选择使用合理的微分方程数值解法,从而得到随着时间变化高压油管内平均压强的变化。龙格-库塔算法是一种流行的高精度数值方法,最常用的为经典四阶龙格-库塔算法($classical\ fourth-order\ RK\ method$),其形式为:
\[
y_{n+1}=y_{n}+\frac{1}{6}(k_{1}+2k_{2}+2k_{3}+k_{4})h
\]

其中:
\begin{align*}
k_{1}&=f(t_{i},y_{i})\\
k_{2}&=f(t_{i}+\frac{1}{2}h,y_{i}+\frac{1}{2}hk_{1})\\
k_{3}&=f(t_{i}+\frac{1}{2}h,y_{i}+\frac{1}{2}hk_{2})\\
k_{4}&=f(t_{i}+h,y_{i}+hk_{3})
\end{align*}

$h$为时间步长

首先,我们任意给定了单向阀的开启时间$\widetilde{t}$,将四阶龙格库塔用于公式(5)求解出压强的数值解,可以得到高压油管、单向阀与喷油嘴组成的系统平衡后





\subsubsection{问题二模型的解决}

 
\subsection{问题三模型}

\subsection{问题四模型}


\section{模型的改进}


\section{模型的优缺点}
\subsection{模型的优点}

\subsection{模型的缺点}



%参考文献
\begin{thebibliography}{9}%宽度9
 \bibitem{bib:one} 蔡梨萍. 基于MATLAB的柴油机高压喷油过程的模拟计算[D].华中科技大学,2005.
 
 \bibitem{bib:two}李旭林,何勇灵.两相条件下柴油机喷油系统的数学模型[J].内燃机学报,2007(05):428-432.
 
 \bibitem{bib:three}Lino, Paolo \& Maione, Bruno \& Rizzo, Alessandro. (2005). A control-oriented model of a Common Rail injection system for diesel engines. IEEE International Conference on Emerging Technologies and Factory Automation, ETFA. 1. 7 pp. - 563. 10.1109/ETFA.2005.1612572. 
 \bibitem{bib:four}卓金武,李必文,魏永生,秦健,MATLAB 在数学建模中的应用,北京:北京 航空航天大学出版社,2014。

\end{thebibliography}

\newpage
%附录
\appendix
\section{主程序--matlab 源程序}
\begin{lstlisting}[language=matlab]
function [dist,xiane,xinyuzhi] = clac()
info_renwu = load('renwu_information.txt');
row = size(info_renwu,1);
col = size(info_renwu,2);
dist = zeros(row,1);
xiane  = zeros(row,1);
xinyuzhi = zeros(row,1);
dingjia = info_renwu(:,3);
for i = 1:row
[dist(i),xiane(i),xinyuzhi(i)]=variable(info_renwu,i); 
end

 \end{lstlisting}

\section{计算距离--matlab 源程序}
\begin{lstlisting}[language=matlab]
function dis = distance(wei1,jing1,wei2,jing2)
R = 6370;
%dis = asin(sqrt(sin((wei1-wei2)/2)^2+cos(wei1)*cos(wei2)*sin((jing1-jing2)/2)^2));
%dis = dis*2*R;
dis = 2*pi*R/360*((wei1-wei2)^2+(jing1-jing2)^2*cos((wei1+wei2)/2)^2)^0.5;

\section{聚类--matlab 源程序}
\begin{lstlisting}[language=matlab]
function [dist,xiane,xinyuzhi] = variable(info_renwu,i)
info_huiyuan = load('huiyuan_location.txt');
data_xiane = info_huiyuan(:,3);
data_xinyu = info_huiyuan(:,4);
data_wei = info_huiyuan(:,1);
data_jing = info_huiyuan(:,2);
num = size(data_jing,1);
data_distance = zeros(num,1);
sum = 0;
for j = 1:num
data_distance(j) = distance(info_renwu(i,1),info_renwu(i,2),info_huiyuan(j,1),info_huiyuan(j,2));
end
[a,b] = sort(data_distance);
first = b(1:30,:);
for count = 1:30
sum = sum + data_distance(first(count));
end
dist = sum/30;
sum = 0;
for count = 1:30
sum = sum + data_xiane(first(count));
end
xiane = sum/30;
sum = 0;
for count = 1:30
sum = sum + data_xinyu(first(count));
end
xinyuzhi = sum/30;

\end{lstlisting}






\end{document} 
